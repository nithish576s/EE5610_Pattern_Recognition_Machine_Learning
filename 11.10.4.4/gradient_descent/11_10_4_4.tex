\documentclass[journal,12pt,twocolumn]{IEEEtran}
%
\usepackage{setspace}
\usepackage{gensymb}
%\doublespacing
\singlespacing

%\usepackage{graphicx}
%\usepackage{amssymb}
%\usepackage{relsize}
\usepackage[cmex10]{amsmath}
%\usepackage{amsthm}
%\interdisplaylinepenalty=2500
%\savesymbol{iint}
%\usepackage{txfonts}
%\restoresymbol{TXF}{iint}
%\usepackage{wasysym}
\usepackage{amsthm}
%\usepackage{iithtlc}
\usepackage{mathrsfs}
\usepackage{txfonts}
\usepackage{stfloats}
\usepackage{bm}
\usepackage{cite}
\usepackage{cases}
\usepackage{subfig}
%\usepackage{xtab}
\usepackage{longtable}
\usepackage{multirow}
%\usepackage{algorithm}
%\usepackage{algpseudocode}
\usepackage{enumitem}
\usepackage{mathtools}
\usepackage{steinmetz}
\usepackage{tikz}
\usepackage{circuitikz}
\usepackage{verbatim}
\usepackage{tfrupee}
\usepackage[breaklinks=true]{hyperref}
%\usepackage{stmaryrd}
\usepackage{tkz-euclide} % loads  TikZ and tkz-base
%\usetkzobj{all}
\usetikzlibrary{calc,math}
\usepackage{listings}
    \usepackage{color}                                            %%
    \usepackage{array}                                            %%
    \usepackage{longtable}                                        %%
    \usepackage{calc}                                             %%
    \usepackage{multirow}                                         %%
    \usepackage{hhline}                                           %%
    \usepackage{ifthen}                                           %%
  %optionally (for landscape tables embedded in another document): %%
    \usepackage{lscape}     
\usepackage{multicol}
\usepackage{chngcntr}
%\usepackage{enumerate}

%\usepackage{wasysym}
%\newcounter{MYtempeqncnt}
\DeclareMathOperator*{\Res}{Res}
%\renewcommand{\baselinestretch}{2}
\renewcommand\thesection{\arabic{section}}
\renewcommand\thesubsection{\thesection.\arabic{subsection}}
\renewcommand\thesubsubsection{\thesubsection.\arabic{subsubsection}}

\renewcommand\thesectiondis{\arabic{section}}
\renewcommand\thesubsectiondis{\thesectiondis.\arabic{subsection}}
\renewcommand\thesubsubsectiondis{\thesubsectiondis.\arabic{subsubsection}}

% correct bad hyphenation here
\hyphenation{op-tical net-works semi-conduc-tor}
\def\inputGnumericTable{}                                 %%

\lstset{
%language=C,
frame=single, 
breaklines=true,
columns=fullflexible
}
%\lstset{
%language=tex,
%frame=single, 
%breaklines=true
%}


\begin{document}
%


\newtheorem{theorem}{Theorem}[section]
\newtheorem{problem}{Problem}
\newtheorem{proposition}{Proposition}[section]
\newtheorem{lemma}{Lemma}[section]
\newtheorem{corollary}[theorem]{Corollary}
\newtheorem{example}{Example}[section]
\newtheorem{definition}[problem]{Definition}
%\newtheorem{thm}{Theorem}[section] 
%\newtheorem{defn}[thm]{Definition}
%\newtheorem{algorithm}{Algorithm}[section]
%\newtheorem{cor}{Corollary}
\newcommand{\BEQA}{\begin{eqnarray}}
\newcommand{\EEQA}{\end{eqnarray}}
\newcommand{\define}{\stackrel{\triangle}{=}}

\bibliographystyle{IEEEtran}
%\bibliographystyle{ieeetr}


\providecommand{\mbf}{\mathbf}
\providecommand{\pr}[1]{\ensuremath{\Pr\left(#1\right)}}
\providecommand{\qfunc}[1]{\ensuremath{Q\left(#1\right)}}
\providecommand{\sbrak}[1]{\ensuremath{{}\left[#1\right]}}
\providecommand{\lsbrak}[1]{\ensuremath{{}\left[#1\right.}}
\providecommand{\rsbrak}[1]{\ensuremath{{}\left.#1\right]}}
\providecommand{\brak}[1]{\ensuremath{\left(#1\right)}}
\providecommand{\lbrak}[1]{\ensuremath{\left(#1\right.}}
\providecommand{\rbrak}[1]{\ensuremath{\left.#1\right)}}
\providecommand{\cbrak}[1]{\ensuremath{\left\{#1\right\}}}
\providecommand{\lcbrak}[1]{\ensuremath{\left\{#1\right.}}
\providecommand{\rcbrak}[1]{\ensuremath{\left.#1\right\}}}
\theoremstyle{remark}
\newtheorem{rem}{Remark}
\newcommand{\sgn}{\mathop{\mathrm{sgn}}}
\providecommand{\abs}[1]{\left\vert#1\right\vert}
\providecommand{\res}[1]{\Res\displaylimits_{#1}} 
\providecommand{\norm}[1]{\left\lVert#1\right\rVert}
%\providecommand{\norm}[1]{\lVert#1\rVert}
\providecommand{\mtx}[1]{\mathbf{#1}}
\providecommand{\mean}[1]{E\left[ #1 \right]}
\providecommand{\fourier}{\overset{\mathcal{F}}{ \rightleftharpoons}}
%\providecommand{\hilbert}{\overset{\mathcal{H}}{ \rightleftharpoons}}
\providecommand{\system}{\overset{\mathcal{H}}{ \longleftrightarrow}}
	%\newcommand{\solution}[2]{\textbf{Solution:}{#1}}
\newcommand{\solution}{\noindent \textbf{Solution: }}
\newcommand{\cosec}{\,\text{cosec}\,}
\providecommand{\dec}[2]{\ensuremath{\overset{#1}{\underset{#2}{\gtrless}}}}
\newcommand{\myvec}[1]{\ensuremath{\begin{pmatrix}#1\end{pmatrix}}}
\newcommand{\mydet}[1]{\ensuremath{\begin{vmatrix}#1\end{vmatrix}}}
%\numberwithin{equation}{section}
\numberwithin{equation}{subsection}
%\numberwithin{problem}{section}
%\numberwithin{definition}{section}
\makeatletter
\@addtoreset{figure}{problem}
\makeatother

\let\StandardTheFigure\thefigure
\let\vec\mathbf
%\renewcommand{\thefigure}{\theproblem.\arabic{figure}}
\renewcommand{\thefigure}{\theproblem}
%\setlist[enumerate,1]{before=\renewcommand\theequation{\theenumi.\arabic{equation}}
%\counterwithin{equation}{enumi}


%\renewcommand{\theequation}{\arabic{subsection}.\arabic{equation}}

\def\putbox#1#2#3{\makebox[0in][l]{\makebox[#1][l]{}\raisebox{\baselineskip}[0in][0in]{\raisebox{#2}[0in][0in]{#3}}}}
     \def\rightbox#1{\makebox[0in][r]{#1}}
     \def\centbox#1{\makebox[0in]{#1}}
     \def\topbox#1{\raisebox{-\baselineskip}[0in][0in]{#1}}
     \def\midbox#1{\raisebox{-0.5\baselineskip}[0in][0in]{#1}}

\vspace{3cm}


\title{Optimization}
\author{S Nithish}
% make the title area
\maketitle

\newpage

%\tableofcontents

\bigskip

\renewcommand{\thefigure}{\theenumi}
\renewcommand{\thetable}{\theenumi}
%\renewcommand{\theequation}{\theenumi}


\begin{abstract}
This document contains the solution of the question from NCERT 11th standard chapter 10 exercise 10.4 problem 4
\end{abstract}

%Download all python codes 
%
%\begin{lstlisting}
%svn co https://github.com/JayatiD93/trunk/My_solution_design/codes
%\end{lstlisting}

%Download all and latex-tikz codes from 
%
%\begin{lstlisting}
%svn co https://github.com/gadepall/school/trunk/ncert/geometry/figs
%\end{lstlisting}
%
\section{Exercise 10.4}
\begin{enumerate}
	\item What are the points on y axis whose distance from the line $\frac{x}{3}+\frac{y}{4}=1$ is 4 units.
	
		The given line is,

		\begin{align}
			\myvec{4 & 3}\vec{x} = 12
		\end{align}

		Let the required point on y axis be $\myvec{0\\y}$, then the distance of this point from the given line is,   
		\begin{align}
			d &= \frac{\abs{\myvec{4 & 3}\myvec{0\\y}-12}}{\sqrt{3^2+4^2}}\\
			d &= \frac{\abs{3y-12}}{5}\\
			d &= 4 \implies \frac{\abs{3y-12}}{5} = 4\\
			\abs{3y-12} &= 20\\
			y &= 4 + \frac{20}{3}=\frac{32}{3} \text{or} y = 4 - \frac{20}{3}=-\frac{8}{3}\\
		\end{align}

The foot of perpendicular to the line from the point $(0,\frac{32}{3})$ is,

		\begin{align}
			\vec{x}_0 = \min_{\vec{x}} \norm{\vec{x}-\myvec{0\\\frac{32}{3}}}^2\\
			\text{s.t} \quad \myvec{4 & 3}\vec{x} = 12
		\end{align}

This can be coverted into an unconstrained optimization problem,
	
		\begin{align}
			\vec{x}_0 = \min_{\vec{\lambda}} \norm{\lambda \myvec{3\\-4}+\myvec{0\\4}-\myvec{0\\\frac{32}{3}}}^2\\
		\end{align}

		\begin{align}
			\norm{\lambda \myvec{3\\-4}+\myvec{0\\4}-\myvec{0\\\frac{32}{3}}}^2&= \norm{\lambda \myvec{3\\-4}+\myvec{0\\-\frac{20}{3}}}^2\\
			= 25\lambda^2 &+2\lambda \brak{\frac{80}{3}}+\brak{\frac{20}{3}}^2
		\end{align}

A numerical solution for this can be obtained with,
		\begin{align}
			\lambda_{n+1} = \lambda_n - \alpha f'(\lambda_n)
		\end{align}

Here $\lambda_0$ is the initial guess of lambda and $\alpha$ is the step size in gradient descent.

		\begin{align}
			f'(\lambda_n) = 50\lambda_n+\frac{160}{3}\\
		\end{align}
So we get,
		\begin{align}
			\lambda_{n+1} = \lambda_n - \alpha \brak{50\lambda+\frac{160}{3}} 
		\end{align}

We get the optimal $\lambda$ to be,
		\begin{align}
			\lambda^* = -1.0667
		\end{align}
\begin{align}
	\vec{x}_0 &= \myvec{3\lambda\\4\brak{1-\lambda}}\\
		  &=\myvec{-3.2\\8.2667}
\end{align}

		\begin{table}[h]
			\centering
			\input{table/11_10_4_4_table1.tex}
			\caption{}
			\label{tab:1}
		\end{table}
		
The foot of perpendicular to the line from the point $(0,-\frac{8}{3})$ is,

		\begin{align}
			\vec{x}_0 = \min_{\vec{x}} \norm{\vec{x}-\myvec{0\\-\frac{8}{3}}}^2\\
			\text{s.t} \quad \myvec{4 & 3}\vec{x} = 12
		\end{align}

This can be coverted into an unconstrained optimization problem,
	
		\begin{align}
			\vec{x}_0 = \min_{\vec{\lambda}} \norm{\lambda \myvec{3\\-4}+\myvec{0\\4}-\myvec{0\\-\frac{8}{3}}}^2\\
		\end{align}

		\begin{align}
			\norm{\lambda \myvec{3\\-4}+\myvec{0\\4}-\myvec{0\\-\frac{8}{3}}}^2&= \norm{\lambda \myvec{3\\-4}+\myvec{0\\\frac{20}{3}}}^2\\
			= 25\lambda^2 &-2\lambda \brak{\frac{80}{3}}+\brak{\frac{20}{3}}^2
		\end{align}

A numerical solution for this can be obtained with,
		\begin{align}
			\lambda_{n+1} = \lambda_n - \alpha f'(\lambda_n)
		\end{align}

Here $\lambda_0$ is the initial guess of lambda and $\alpha$ is the step size in gradient descent.

		\begin{align}
			f'(\lambda_n) = 50\lambda_n-\frac{160}{3}\\
		\end{align}
So we get,
		\begin{align}
			\lambda_{n+1} = \lambda_n - \alpha \brak{50\lambda-\frac{160}{3}} 
		\end{align}

We get the optimal $\lambda$ to be,

		\begin{align}
			\lambda^* = 1.0667
		\end{align}
\begin{align}
	\vec{x}_1 = \myvec{3.2\\-0.2667}
\end{align}
		\begin{table}[h]
			\centering
			%%%%%%%%%%%%%%%%%%%%%%%%%%%%%%%%%%%%%%%%%%%%%%%%%%%%%%%%%%%%%%%%%%%%%%
%%                                                                  %%
%%  This is a LaTeX2e table fragment exported from Gnumeric.        %%
%%                                                                  %%
%%%%%%%%%%%%%%%%%%%%%%%%%%%%%%%%%%%%%%%%%%%%%%%%%%%%%%%%%%%%%%%%%%%%%%

\begin{center}
\begin{tabular}{|c|c|c|}
\hline
	\textbf{Parameter}& \textbf{Value}& \textbf{Description}\\ \hline
	$\lambda_0$	&$2$		        &Initial guess\\ \hline
	$\alpha$	&$0.01$	                &step size\\ \hline
	$N$	        &$10000$                &Number of iterations\\ \hline
	$\epsilon$	&$10^{-7}/$             &Tolerance in $\lambda$\\ \hline
\end{tabular}
\end{center}

			\caption{}
			\label{tab:2}
		\end{table}

\end{enumerate}

%\begin{enumerate}[label=\thesection.\arabic*.,ref=\thesection.\theenumi]
%\num then what are its direction cosines ?berwithin{equation}{enumi}
%\item Verification of the above problem using python code.\\
%%\solution The  following Python code generates Fig. \ref{fig:point_distance}
%%\begin{lstlisting}
%%codes/det_check.py
%%\end{lstlisting}
%
%\end{enumerate}

\end{document}

