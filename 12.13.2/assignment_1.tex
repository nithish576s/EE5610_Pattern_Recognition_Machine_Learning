\documentclass[journal,12pt,twocolumn]{IEEEtran}
%
\usepackage{setspace}
\usepackage{gensymb}
%\doublespacing
\singlespacing

%\usepackage{graphicx}
%\usepackage{amssymb}
%\usepackage{relsize}
\usepackage[cmex10]{amsmath}
%\usepackage{amsthm}
%\interdisplaylinepenalty=2500
%\savesymbol{iint}
%\usepackage{txfonts}
%\restoresymbol{TXF}{iint}
%\usepackage{wasysym}
\usepackage{amsthm}
%\usepackage{iithtlc}
\usepackage{mathrsfs}
\usepackage{txfonts}
\usepackage{stfloats}
\usepackage{bm}
\usepackage{cite}
\usepackage{cases}
\usepackage{subfig}
%\usepackage{xtab}
\usepackage{longtable}
\usepackage{multirow}
%\usepackage{algorithm}
%\usepackage{algpseudocode}
\usepackage{enumitem}
\usepackage{mathtools}
\usepackage{steinmetz}
\usepackage{tikz}
\usepackage{circuitikz}
\usepackage{verbatim}
\usepackage{tfrupee}
\usepackage[breaklinks=true]{hyperref}
%\usepackage{stmaryrd}
\usepackage{tkz-euclide} % loads  TikZ and tkz-base
%\usetkzobj{all}
\usetikzlibrary{calc,math}
\usepackage{listings}
    \usepackage{color}                                            %%
    \usepackage{array}                                            %%
    \usepackage{longtable}                                        %%
    \usepackage{calc}                                             %%
    \usepackage{multirow}                                         %%
    \usepackage{hhline}                                           %%
    \usepackage{ifthen}                                           %%
  %optionally (for landscape tables embedded in another document): %%
    \usepackage{lscape}     
\usepackage{multicol}
\usepackage{chngcntr}
%\usepackage{enumerate}

%\usepackage{wasysym}
%\newcounter{MYtempeqncnt}
\DeclareMathOperator*{\Res}{Res}
%\renewcommand{\baselinestretch}{2}
\renewcommand\thesection{\arabic{section}}
\renewcommand\thesubsection{\thesection.\arabic{subsection}}
\renewcommand\thesubsubsection{\thesubsection.\arabic{subsubsection}}

\renewcommand\thesectiondis{\arabic{section}}
\renewcommand\thesubsectiondis{\thesectiondis.\arabic{subsection}}
\renewcommand\thesubsubsectiondis{\thesubsectiondis.\arabic{subsubsection}}

% correct bad hyphenation here
\hyphenation{op-tical net-works semi-conduc-tor}
\def\inputGnumericTable{}                                 %%

\lstset{
%language=C,
frame=single, 
breaklines=true,
columns=fullflexible
}
%\lstset{
%language=tex,
%frame=single, 
%breaklines=true
%}


\begin{document}
%


\newtheorem{theorem}{Theorem}[section]
\newtheorem{problem}{Problem}
\newtheorem{proposition}{Proposition}[section]
\newtheorem{lemma}{Lemma}[section]
\newtheorem{corollary}[theorem]{Corollary}
\newtheorem{example}{Example}[section]
\newtheorem{definition}[problem]{Definition}
%\newtheorem{thm}{Theorem}[section] 
%\newtheorem{defn}[thm]{Definition}
%\newtheorem{algorithm}{Algorithm}[section]
%\newtheorem{cor}{Corollary}
\newcommand{\BEQA}{\begin{eqnarray}}
\newcommand{\EEQA}{\end{eqnarray}}
\newcommand{\define}{\stackrel{\triangle}{=}}

\bibliographystyle{IEEEtran}
%\bibliographystyle{ieeetr}


\providecommand{\mbf}{\mathbf}
\providecommand{\pr}[1]{\ensuremath{\Pr\left(#1\right)}}
\providecommand{\qfunc}[1]{\ensuremath{Q\left(#1\right)}}
\providecommand{\sbrak}[1]{\ensuremath{{}\left[#1\right]}}
\providecommand{\lsbrak}[1]{\ensuremath{{}\left[#1\right.}}
\providecommand{\rsbrak}[1]{\ensuremath{{}\left.#1\right]}}
\providecommand{\brak}[1]{\ensuremath{\left(#1\right)}}
\providecommand{\lbrak}[1]{\ensuremath{\left(#1\right.}}
\providecommand{\rbrak}[1]{\ensuremath{\left.#1\right)}}
\providecommand{\cbrak}[1]{\ensuremath{\left\{#1\right\}}}
\providecommand{\lcbrak}[1]{\ensuremath{\left\{#1\right.}}
\providecommand{\rcbrak}[1]{\ensuremath{\left.#1\right\}}}
\theoremstyle{remark}
\newtheorem{rem}{Remark}
\newcommand{\sgn}{\mathop{\mathrm{sgn}}}
\providecommand{\abs}[1]{\left\vert#1\right\vert}
\providecommand{\res}[1]{\Res\displaylimits_{#1}} 
\providecommand{\norm}[1]{\left\lVert#1\right\rVert}
%\providecommand{\norm}[1]{\lVert#1\rVert}
\providecommand{\mtx}[1]{\mathbf{#1}}
\providecommand{\mean}[1]{E\left[ #1 \right]}
\providecommand{\fourier}{\overset{\mathcal{F}}{ \rightleftharpoons}}
%\providecommand{\hilbert}{\overset{\mathcal{H}}{ \rightleftharpoons}}
\providecommand{\system}{\overset{\mathcal{H}}{ \longleftrightarrow}}
	%\newcommand{\solution}[2]{\textbf{Solution:}{#1}}
\newcommand{\solution}{\noindent \textbf{Solution: }}
\newcommand{\cosec}{\,\text{cosec}\,}
\providecommand{\dec}[2]{\ensuremath{\overset{#1}{\underset{#2}{\gtrless}}}}
\newcommand{\myvec}[1]{\ensuremath{\begin{pmatrix}#1\end{pmatrix}}}
\newcommand{\mydet}[1]{\ensuremath{\begin{vmatrix}#1\end{vmatrix}}}
%\numberwithin{equation}{section}
\numberwithin{equation}{subsection}
%\numberwithin{problem}{section}
%\numberwithin{definition}{section}
\makeatletter
\@addtoreset{figure}{problem}
\makeatother

\let\StandardTheFigure\thefigure
\let\vec\mathbf
%\renewcommand{\thefigure}{\theproblem.\arabic{figure}}
\renewcommand{\thefigure}{\theproblem}
%\setlist[enumerate,1]{before=\renewcommand\theequation{\theenumi.\arabic{equation}}
%\counterwithin{equation}{enumi}


%\renewcommand{\theequation}{\arabic{subsection}.\arabic{equation}}

\def\putbox#1#2#3{\makebox[0in][l]{\makebox[#1][l]{}\raisebox{\baselineskip}[0in][0in]{\raisebox{#2}[0in][0in]{#3}}}}
     \def\rightbox#1{\makebox[0in][r]{#1}}
     \def\centbox#1{\makebox[0in]{#1}}
     \def\topbox#1{\raisebox{-\baselineskip}[0in][0in]{#1}}
     \def\midbox#1{\raisebox{-0.5\baselineskip}[0in][0in]{#1}}

\vspace{3cm}


\title{Assignment 1}
\author{S Nithish}





% make the title area
\maketitle

\newpage

%\tableofcontents

\bigskip

\renewcommand{\thefigure}{\theenumi}
\renewcommand{\thetable}{\theenumi}
%\renewcommand{\theequation}{\theenumi}


\begin{abstract}
This document contains the questions of NCERT class 12 chapter 13 exercise 13.2
\end{abstract}

%Download all python codes 
%
%\begin{lstlisting}
%svn co https://github.com/JayatiD93/trunk/My_solution_design/codes
%\end{lstlisting}

%Download all and latex-tikz codes from 
%
%\begin{lstlisting}
%svn co https://github.com/gadepall/school/trunk/ncert/geometry/figs
%\end{lstlisting}
%


\section{Exercise 13.2}
\begin{enumerate}

\item If $P(A)=\frac{3}{5} \text{ and } P(B)=\frac{1}{5}$, find $P(A\cap B)$ if $A$ and $B$ are independent events.

\item Two cards are drawn at random and without replacement from a pack of 52
playing cards. Find the probability that both the cards are black.

\item A box of oranges is inspected by examining three randomly selected oranges drawn without replacement. If all the three oranges are good, the box is approved for sale, otherwise, it is rejected. Find the probability that a box containing 15 oranges out of which 12 are good and 3 are bad ones will be approved for sale.

\item  A fair coin and an unbiased die are tossed. Let $A$ be the event 'head appears on the coin' and $B$ be the event '3 on the die'. Check whether $A$ and $B$ are independent events or not.

\item  A die marked 1, 2, 3 in red and 4, 5, 6 in green is tossed. Let $A$ be the event, 'the number is even, ' and $B$ be the event, 'the number is red'. Are $A$ and $B$ independent?

\item Let $E$ and $F$ be events with $P(E)=\frac{3}{5}$, $P(F)=\frac{3}{10}$ and $P(E \cap F)=\frac{1}{5}$. Are $E$ and $F$ independent?

\item Given that the events A and B are such that $P(A)=\frac{1}{2}$, $P(A \cup B)=\frac{3}{5}$ and $P(B)=p$. Find p if they are

\begin{enumerate}
\item mutually exclusive
\item independent
\end{enumerate}

\item Let $A$ and $B$ be independent events with $P(A)=0.3$ and $P(B)=0.4$. Find

\begin{enumerate}
\item $P(A \cap B)$
\item $P(A \cup B)$
\item $P(A|B)$
\item $P(B|A)$
\end{enumerate}

\item If A and B are two events such that $P(A)=\frac{1}{4}$, $P(B)=\frac{1}{2}$ and $P(A \cap B)=\frac{1}{8}$, find $P(\text{not }A \text{ and not }B)$

\item Events $A$ and $B$ are such that $P(A)=\frac{1}{2}$, $P(B)=\frac{7}{12}$ and $P(\text{not }A \text{ or not }B)=\frac{1}{4}$. State whether $A$ and $B$ are independent?

\item Given two independent events A and B such that $P(A) = 0.3$, $P(B) = 0.6$. Find

\begin{enumerate}
\item $P(A\text{ and } B)$
\item $P(A \text{ and not } B)$
\item $P(A \text{ or } B)$
\item $P(\text{neither } A \text{ nor } B)$

\end{enumerate}


\item A die is tossed thrice. Find the probability of getting an odd number at least once.

\item Two balls are drawn at random with replacement from a box containing 10 black and 8 red balls. Find the probability that

\begin{enumerate}
\item both balls are red.
\item first ball is black and second is red.
\item one of them is black and other is red.

\end{enumerate}

\item Probability of solving specific problem independently by A and B are $\frac{1}{2}$ and $\frac{1}{3}$ respectively. If both try to solve the problem independently, find the probability that

\begin{enumerate}
\item the problem is solved
\item exactly one of them solves the problem
\end{enumerate}

\item One card is drawn at random from a well shuffled deck of 52 cards. In which of the following cases are the events $E$ and $F$ independent ?

\begin{enumerate}
\item $E$: 'the card drawn is spade'\\
$F$: 'the card drawn is an ace'

\item $E$: 'the card drawn is black’\\
$F$ : 'the card drawn is a king’

\item $E$ : 'the card drawn is a king or queen’\\
$F$ : 'the card drawn is a queen or jack’
\end{enumerate}

\item In a hostel, 60\% of the students read Hindi newspaper, 40\% read English newspaper and 20\% read both Hindi and English newspapers. A student is selected at random.

\begin{enumerate}
\item Find the probability that she reads neither Hindi nor English newspapers.

\item If she reads Hindi newspaper, find the probability that she reads English newspaper.

\item If she reads English newspaper, find the probability that she reads Hindi newspaper.\\
\end{enumerate}

\end{enumerate}

Choose the correct answer in Exercises 17 and 18.

\begin{enumerate}[resume]

\item The probability of obtaining an even prime number on each die, when a pair of dice is rolled is

\begin{enumerate}
\item 0
\item $\frac{1}{3}$
\item $\frac{1}{12}$
\item $\frac{1}{36}$
\end{enumerate}

\item Two events $A$ and $B$ will be independent, if

\begin{enumerate}
\item $A$ and $B$ are mutually exclusive
\item P(not $A$ $\cap$ not $B$) = $\sbrak{1-P(A)}\sbrak{1-P(B)} $
\item $P(A) = P(B)$
\item $P(A) + P(B) = 1$
\end{enumerate}

\end{enumerate}





%\begin{enumerate}[label=\thesection.\arabic*.,ref=\thesection.\theenumi]
%\numberwithin{equation}{enumi}
%\item Verification of the above problem using python code.\\
%%\solution The  following Python code generates Fig. \ref{fig:point_distance}
%%\begin{lstlisting}
%%codes/det_check.py
%%\end{lstlisting}
%
%\end{enumerate}

\end{document}



